\documentclass[10pt,landscape,a4paper]{article}
\usepackage[utf8]{inputenc}
\usepackage{ctex}
\usepackage{graphicx}
\usepackage{tabularx}
\usepackage{graphics}
\usepackage{enumitem}
\usepackage{tikz}
\usetikzlibrary{shapes,positioning,arrows,fit,calc,graphs,graphs.standard}
\usepackage[nosf]{kpfonts}
\usepackage[t1]{sourcesanspro}
\usepackage{multicol}
\usepackage{wrapfig}
\usepackage[top=2mm,bottom=2mm,left=2mm,right=2mm]{geometry}
\usepackage[framemethod=tikz]{mdframed}
\usepackage{pdfpages}
\usepackage{cancel}


\let\bar\overline
\newfontface\CJKfont{SimSun}
\setlength{\columnsep}{10pt} % 设置栏间距
\setlength{\columnseprule}{0.2pt}
\setlength{\parskip}{2pt}
\setlist{topsep=0em,itemsep=-0.5em, labelsep=0em, leftmargin=1.5em}

\sloppy
\begin{document}
	
	\begin{multicols*}{3}
	
	
	注意,这是一个同轨学习方法,因为产生样本的策略是$\pi\left(a\mid s,\theta_t\right)$,而策略改进正是为了改进这一策略。
	
	第三个改进是通过重要性采样使得A2C具备离轨学习能力。重要性采样是指通过从一个行为分布采样得到的数据,对服从另一个目标分布的随机变量进行估计。例如,已知一个目标分布T,一个行为分布B,则服从目标分布的随机变量X的数学期望:
	
	$E_{X\sim T}[X]=\sum_x p_T(x) x=\sum_x p_B(x)\frac{p_T(x)}{p_B(x)} x=E_{X\sim B}\left[\frac{p_T(x)}{p_B(x)} X\right]\approx\frac{1}{n}\sum_{i=1}^n\frac{p_T\left(x_i\right)}{p_B\left(x_i\right)} x_i$
	
	可见,通过对从行为分布采样得到的数据进行加权,即可实现对服从目标分布的随机变量数学期望的估计。这里,目标分布与行为分布在给定值处概率质量/概率密度之比,称为该值的重要性权重。而应用重要性采样估计数学期望时,应利用重要性权重计算加权平均。
	
	因此,在离轨学习中,如果数据来自于行为策略$\beta$而非目标策略$\pi$,则策略梯度的计算应遵循离轨策略梯度定理,即:
	
	$\nabla_{\theta} J(\theta)=E_{S\sim\eta, A\sim\beta}\left[\frac{\pi(A\mid S,\theta)}{\beta(A\mid S)}\nabla_{\theta}\ln\pi(A\mid S,\theta) q_{\pi}(S, A)\right]$
	
	简言之,应该将同轨学习中策略对数的梯度乘以重要性权重$\pi(A|S,\theta)/\beta(A|S)$,即目标策略与行为策略在给定状态下采取特定行动的概率之比。值得注意的是,离轨策略梯度仍然具有基线不变性,因此可以使用优势函数代替行动价值。
	
	综合上述分析,使用优势函数的离轨A2C算法,其伪代码如下:
	
	注意其中加入的重要性权重。
	\end{multicols*}
\end{document}