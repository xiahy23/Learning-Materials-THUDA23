\section{Algebraische Strukturen}
\subsection*{Monoid}
Ein Monoid ist ein Tupel $(M,*,e)$ bestehend aus einer Menge $M$, einer zweistelligen
Verknüpfung $*:M\times M\to M,\quad (a,b)\mapsto a*b$ und einem neutralem Element $e\in M$.
Außerdem muss gelten:\\
Assoziativität:\\
(M1) $\forall a,b,c\in M:(a*b)*c=a*(b*c)$\\
neutrales Element:\\
(M2) $\forall a\in M:e*a=a*e=a$\\
\emph{Beispiel:}\\
$(\mathbb{N}_0,+,0)$ ist Monoid\\
$(\mathbb{N},\cdot,1)$ ist Monoid
\subsection*{Gruppe}
Eine Gruppe ist ein Tupel $(G,*,e,a^{-1})$ bestehend aus einer Menge $G$, einer zweistelligen Verknüpfung $*:G\times G\to G,\quad (a,b)\mapsto a*b$, einem neutralem Element $e\in G$ und einem inversen Element $a^{-1}\in G$.
Außerdem muss gelten:\\
Abgeschlossenheit:\\
(G1) $\forall a,b\in G:(a*b)\in G$\\
Assoziativität:\\
(G2) $\forall a,b,c\in G:(a*b)*c=a*(b*c)$\\
neutrales Element:\\
(G3) $\forall a\in G:e*a=a*e=a$\\
inverses Element:\\
(G4) $\forall a\in G\exists a^{-1}\in G:a*a^{-1}=a^{-1}*a=e$\\
Für abelsche Gruppe:\\
Kommutativität:\\
(G5) $\forall a,b\in G:a*b=b*a$\\
\emph{Beispiel:}\\
$(\mathbb{Z},+,0,-a)$ ist Gruppe\\
$(\mathbb{Q}\setminus\{0\},\cdot,1,\frac{1}{a})$ ist Gruppe
\subsection*{Körper}
Ein Körper ist ein Tupel $(K,+,\cdot)$, bestehend aus einer Menge $K$ und zwei 
zweistelligen Operationen $+$ und $\cdot$ auf $K$.\\
Es muss gelten:\\
(K1) $(K,+)$ ist abelsche Gruppe mit neutralem Element $0$.\\
(K2) $(K\setminus \{0\},\cdot)$ ist abelsche Gruppe mit neutralem Element $1$.\\
(K3) Distributivität:\\
$\forall a,b,c\in K:a\cdot (b+c)=a\cdot b+a\cdot c$\\
\emph{Beispiel:}\\
$(\mathbb{Q},+,\cdot)$ ist Körper\\
$(\mathbb{R},+,\cdot)$ ist Körper
%\subsection*{Schaltfunktionen}
%\subsection*{Boolsche Algebra}